\section{Finding similar movies}

\subsection{Jaccard similarity}
Movie similarity is defined as Jaccard similarity of sets of a subset of movie attributes. Given two sets \textit{A} and \textit{B} the Jaccard similarity coefficient, \textit{j}, is defined as \(j = \frac{|A \cap B|}{|A \cup B|}\). The attributes considered when measuring similarity is title, genres, actors and directors. A set of actors is the identifiers of the actors having roles in the movie, a set of directors is the identifiers of the directors, and a set of genres is the names of the genres of the movie. The set that movie similarity is measured from is the union of all these sets as well as the title of the movie and is denoted \textit{C}.


\subsection{Min Hashing}
The purpose of Min Hashing is reducing the size the comparable movie representation, such that it still preserves its Jaccard similarity when being compared to other MinHashed objects, as well as making the representation suitable for locality-sensitive hashing.\\ \\
Min Hashing transforms \textit{C} to a signature, \(S\), which contains an integer for each hashing function used. The signature is generated by applying \textit{n} hash functions \(h_1(C), h_2(C),\ \dots\ , h_n(C)\) to the set, which produces a vector \([h_1(C), h_2(C),\ \dots\ , h_n(C)]\) of size \textit{n}, each element being the result of a hash function.\\ \\
The hash functions are defined using permutations. A permutation is a random ordering of all possible values of elements of \textit{C}, which is the union of all the sets that are to be compared. From a permutation, the hash value is computed by iterating through each element of the permutation, in the random order of the permutation. When an element is found that exists in \textit{C}, the hash function returns the index of that element in the ordering of the permutation.\\ \\
The reasoning for using Min Hash function to calculate signatures is that given two sets, \(C_1, C_2\), the probability of \(h(C_1) = h(C_2)\) is equal to the Jaccard similarity coefficient.
\begin{equation}
P(h(C_1) = h(C_2)) = \frac{|A \cap B|}{|A \cup B|}
\end{equation}
While Min Hashing produces a signature consuming significantly less space than \(C\), it stochastically preserves similarity.\\ \\
For the representation to be suitable for locality-sensitive hashing, each hash function has to be applied on each \(C_1, C_2,\ \dots\ , C_m\), producing \(m\) vectors \(S_1, S_2,\ \dots\ , S_m\), resulting in a matrix that is the input for the locality-sensitive hashing.

\begin{table}[h]
\begin{tabular}{|l|l|l|l|l|}
\hline
 & \(S_1\) & \(S_2\)  & \(S_3\)  &  \(S_4\) \\ \hline
 \(h_1(C)\) &   714  & 55 & 10034 & 183 \\ \hline
  \(h_2(C)\) &     2 & 5213 & 3921 & 5213 \\ \hline
 \(h_3(C)\) &     8322 & 377  & 475 & 15632 \\ \hline
\end{tabular}
\centering
\caption{In this example, the 714'th element of the permutation of the universal set for \(h_1(S)\) is the first element that the universal set and \(S_1\) had in common, starting from element 0.}
\end{table}