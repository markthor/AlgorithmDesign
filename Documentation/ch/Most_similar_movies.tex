\section{Finding similar movies}

\subsection{Jaccard similarity}
Movie similarity is defined as Jaccard similarity of sets of a subset of movie attributes. Given two sets \textit{A} and \textit{B} the Jaccard similarity coefficient, \textit{j}, is defined as \(j = \frac{|A \cap B|}{|A \cup B|}\). The attributes considered when measuring similarity is title, genres, actors and directors. A set of actors is the identifiers of the actors having roles in the movie, a set of directors is the identifiers of the directors, and a set of genres is the names of the genres of the movie. The set that movie similarity is measured from is the union of all these sets as well as the title of the movie and is denoted \textit{S}.

\subsection{Min Hashing}
....\marginpar{Indsæt når mark er daun}
The purpose of Min Hashing is reducing the size of an attribute set \textit{M} and make the representation of \textit{S} suitable for locality-sensitive hashing. Min Hashing transforms \textit{S} to a signature, which contains an integer for each hashing function used.
The signature is generated by applying\textit{ }n hash functions \(h_1(S), h_2(S),\ \dots\ , h_n(S)\) to the set, which produces a vector \([h_1(S), h_2(S),\  \dots\ , h_n(S)]\) of size \textit{n}, each element being the result of a hash function.
The hash functions are defined using permutations. A permutation is a random ordering of all possible values of elements of \textit{S}, which is the union of all the sets that are to be compared. From a permutation, the hash value is computed by iterating through each element of the permutation, in the random order of the permutation. When an element is found that is in \textit{S}, the hash function returns the index of that element in the ordering of the permutation.
The reasoning for using Min Hash function to calculate signatures is that given two sets, \(S_1, S_2\), the probability of \(h(S_1) = h(S_2)\) is equal to the Jaccard similarity coefficient. While Min Hashing produces a signature consuming significantly less space than \textit{S}, it preserves similarity.
The reason for the Jaccard similarity coefficient being equal to the probability that the Min Hash function returns the same for the compared sets are that the 
